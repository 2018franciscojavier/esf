% Created 2018-04-01 dom 22:37
% Intended LaTeX compiler: pdflatex
\documentclass[xcolor={usenames,svgnames,dvipsnames}]{beamer}
\usepackage[utf8]{inputenc}
\usepackage[T1]{fontenc}
\usepackage{graphicx}
\usepackage{grffile}
\usepackage{longtable}
\usepackage{wrapfig}
\usepackage{rotating}
\usepackage[normalem]{ulem}
\usepackage{amsmath}
\usepackage{textcomp}
\usepackage{amssymb}
\usepackage{capt-of}
\usepackage{hyperref}
\usepackage{color}
\usepackage{listings}
\usepackage{mathpazo}
\usepackage{gensymb}
\usepackage{amsmath}
\usepackage{chemarr}%flechas para reacciones químicas (SFER.tex)
\bibliographystyle{plain}
\AtBeginSubsection[]{\begin{frame}[plain]\tableofcontents[currentsubsection,sectionstyle=show/shaded,subsectionstyle=show/shaded/hide]\end{frame}}
\AtBeginSection[]{\begin{frame}[plain]\tableofcontents[currentsection,hideallsubsections]\end{frame}}
\usepackage[emulate=units]{siunitx}
\sisetup{fraction=nice, decimalsymbol=comma, retain-unity-mantissa = false}
\newunit{\wattpeak}{Wp}
\newunit{\watthour}{Wh}
\newunit{\amperehour}{Ah}
\hypersetup{colorlinks=true, linkcolor=Blue, urlcolor=Blue}
\renewcommand{\thefootnote}{\fnsymbol{footnote}}
\setbeamercolor{alerted text}{fg=blue!50!black} \setbeamerfont{alerted text}{series=\bfseries}
\usetheme[hideothersubsections]{Goettingen}
\usecolortheme{rose}
\usefonttheme{serif}
\author{Oscar Perpiñán Lamigueiro \\ \url{http://oscarperpinan.github.io}}
\date{}
\title{Solar Radiation on a PV Generator}
\subtitle{Fundamentals of PV Engineering}
\hypersetup{
 pdfauthor={Oscar Perpiñán Lamigueiro \\ \url{http://oscarperpinan.github.io}},
 pdftitle={Solar Radiation on a PV Generator},
 pdfkeywords={},
 pdfsubject={},
 pdfcreator={Emacs 25.2.2 (Org mode 9.1.9)}, 
 pdflang={Spanish}}
\begin{document}

\maketitle


\section{Extraterrestrial Irradiation}
\label{sec:org4727159}
\begin{frame}[label={sec:org9c27fcb}]{}
\begin{itemize}
\item \(B_{0}(0)=B_{0}\cdot\epsilon_{0}\cdot\cos\theta_{zs}\)

\item \(B_{0d}(0)=-\frac{24}{\pi}B_{0}\epsilon_{0}\cdot\left(\omega_{s}\sin\phi\sin\delta+\cos\delta\cos\phi\sin\omega_{s}\right)\)
(\(\omega_{s}\) en radianes)
\end{itemize}
\end{frame}

\section{Caracterización de la atmósfera}
\label{sec:org62b3ac3}

\begin{frame}[label={sec:orgdaf22f0}]{Masa de aire}
\begin{itemize}
\item Relación entre camino recorrido por rayos directos del Sol a
través de la atmósfera hasta la superficie receptora y el que
recorrerían en caso de incidencia vertical (AM=1)

\item \(AM=1/\cos\theta_{zs}\)
\end{itemize}
\end{frame}

\begin{frame}[label={sec:org65b315b}]{Índice de claridad}
\begin{itemize}
\item Relación entre la radiación global en el plano horizontal y la
radiación extra-atmosférica en el plano horizontal

\item El índice de claridad \alert{no depende de las variaciones debidas al
movimiento aparente del sol}.

\item \(K_{Tm}=\frac{G_{d,m}(0)}{B_{0d,m}(0)}\) (mensual)
\end{itemize}
\end{frame}

\begin{frame}[label={sec:orge1a6e53}]{Índice de claridad}
\begin{description}
\item[{\(K_{T}\):}] índice de claridad instantáneo. \(K_{T}=G/B_{0}\)

\item[{\(K_{Td}\):}] índice de claridad diario. \(K_{Td}=G_{d}/B_{0d}\)

\item[{\(K_{Tm}\):}] índice de claridad mensual. \(K_{Tm}=G_{m}/B_{0m}=G_{d,m}/B_{0d,m}\)

\item[{\(K_{Ta}\):}] índice de claridad anual. \(K_{Ta} = G_{a}/B_{0a} = \dots\)
\end{description}
\end{frame}
\end{document}