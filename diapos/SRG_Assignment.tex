% Created 2019-02-14 jue 16:29
% Intended LaTeX compiler: pdflatex
\documentclass[11pt]{article}
\usepackage[utf8]{inputenc}
\usepackage[T1]{fontenc}
\usepackage{graphicx}
\usepackage{grffile}
\usepackage{longtable}
\usepackage{wrapfig}
\usepackage{rotating}
\usepackage[normalem]{ulem}
\usepackage{amsmath}
\usepackage{textcomp}
\usepackage{amssymb}
\usepackage{capt-of}
\usepackage{hyperref}
\usepackage{color}
\usepackage{listings}
\usepackage{mathpazo}
\usepackage[margin=2cm]{geometry}
\author{Oscar Perpiñán Lamigueiro}
\date{}
\title{Session \#2 Assignment\\\medskip
\large Fundamentals of Photovoltaic Engineering}
\hypersetup{
 pdfauthor={Oscar Perpiñán Lamigueiro},
 pdftitle={Session \#2 Assignment},
 pdfkeywords={},
 pdfsubject={},
 pdfcreator={Emacs 26.1 (Org mode 9.2)}, 
 pdflang={English}}
\begin{document}

\maketitle
In this assignment you are going to estimate the global irradiation incident on a PV generator mounted on a south-oriented fixed structure with an inclination near to the optimum value. You should use the location and the global radiation values of the step \#8 (or step \#5) of previous assignment.

\begin{enumerate}
\item Compute the declination, \(\delta\), day length, \(w_s\), and daily extraterrestrial irradiation on the horizontal plane, \(B_{0d}(0)\) for the ``average monthly days'' (see table below).

\item Compute the clearness index, \(K_{Tm}\), and fraction of diffuse, \(F_{Dm}\), with the 12 monthly averages of global irradiation on the horizontal plane (step \#8 or \#5 of previous assignment). Compute the beam and diffuse components of solar radiation on the horizontal plane.

\item Compute the cosine of the zenith angle, \(\cos(\theta_{zs})\), and the cosine of the angle of incidence, \(\cos(\theta_s)\) \textbf{for each day (month) of the table}. You should obtain 24 values per day (a value for each hour in a day).

\item Compute the intradaily profiles, \(r_D\) and \(r_G\) \textbf{for each day (month) of the table}. You should obtain 24 values per day (a value for each hour in a day).

\item With the results of steps \#3 and \#4, compute the diffuse, global, and beam hourly irradiation profiles. Therefore, for each day of the table you should obtain 24 values of hourly global radiation (thus, a total of 12·24 = 288 values of hourly global radiation).

\item Transform the hourly beam and diffuse radiation components on the horizontal plane to the plane of the plane of the generator.

\item Sum the results of previous step to obtain the 12 monthly averages of daily global, diffuse, and beam radiation on the plane of the generator.
\end{enumerate}


\begin{center}
\begin{tabular}{lrrrrrrrrrrrr}
Month & Jan & Feb & Mar & Apr & May & Jun & Jul & Aug & Sep & Oct & Nov & Dec\\
\hline
\(d_n\) & 17 & 45 & 74 & 105 & 135 & 161 & 199 & 230 & 261 & 292 & 322 & 347\\
\end{tabular}
\end{center}
\end{document}