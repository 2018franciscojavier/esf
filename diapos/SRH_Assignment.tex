% Created 2018-04-02 lun 17:35
% Intended LaTeX compiler: pdflatex
\documentclass[11pt]{article}
\usepackage[utf8]{inputenc}
\usepackage[T1]{fontenc}
\usepackage{graphicx}
\usepackage{grffile}
\usepackage{longtable}
\usepackage{wrapfig}
\usepackage{rotating}
\usepackage[normalem]{ulem}
\usepackage{amsmath}
\usepackage{textcomp}
\usepackage{amssymb}
\usepackage{capt-of}
\usepackage{hyperref}
\usepackage{color}
\usepackage{listings}
\usepackage{mathpazo}
\author{Oscar Perpiñán Lamigueiro}
\date{}
\title{Session \#1 Assignment\\\medskip
\large Fundamentals of Photovoltaic Engineering}
\hypersetup{
 pdfauthor={Oscar Perpiñán Lamigueiro},
 pdftitle={Session \#1 Assignment},
 pdfkeywords={},
 pdfsubject={},
 pdfcreator={Emacs 25.2.2 (Org mode 9.1.9)}, 
 pdflang={English}}
\begin{document}

\maketitle
\begin{enumerate}
\item Retrieve daily measurements from three nearby meteorological stations (time series length 10 years).
\item \emph{(Session \#2) Filter each time series using physical limits}.
\item Compute a daily time series representative of the region with the average of the three time series. Compare this time series with each station using statistical metrics (MBD, RMSD, MAD).
\item Choose a location inside the perimeter defined by the three stations, and estimate the daily solar radiation using spatial interpolation (IDW).
\item Retrieve monthly averages of solar radiation from a satellite service (preferably CMSAF, with QGis or similar software) for a region covering the three stations.
\item Compare the values at the three locations with the monthly averages of the measurements provided by the stations using statistical metrics.
\item Combine the satellite estimations at the location defined in step 4 with the result of that step, using spatial interpolation (IDW).
\end{enumerate}

\textbf{The result of step 6 will be used in Session \#2}.
\end{document}