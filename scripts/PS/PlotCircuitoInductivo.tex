 % Tensión y corriente en circuito inductivo con PSTricks
 % Copyright (C) 2010 Oscar Perpiñán Lamigueiro
 %
 % This program is free software; you can redistribute it and/or
 % modify it under the terms of the GNU General Public License
 % as published by the Free Software Foundation; either version 2
 % of the License, or (at your option) any later version.
 %
 % This program is distributed in the hope that it will be useful,
 % but WITHOUT ANY WARRANTY; without even the implied warranty of
 % MERCHANTABILITY or FITNESS FOR A PARTICULAR PURPOSE.  See the
 % GNU General Public License for more details.
 %
 % You should have received a copy of the GNU General Public License
 % along with this program; if not, write to the Free Software
 % Foundation, Inc., 59 Temple Place - Suite 330, Boston, MA  02111-1307, USA.


\psset{xunit=8,yunit=3}
\begin{pspicture}(-0.4,-1.2)(1,1.2)
  \psgrid[griddots=5,gridlabels=0pt](-0.2,-1)(1,1)
	\psaxes[Dx=0.2,Ox=0,Oy=0,Dy=0.4,ticks=x,
          axesstyle=frame](0,0)(-0.2,-1)(1,1)
 %\rput(0.5,1.3){Curves example}
 \rput(-0.3,0){\shortstack{%
   \textcolor{blue}{Corriente}\\
      \textcolor{red}{Tensi\'on}}}
  \rput(0.5,-1.3){$t$}
  \psplot[linecolor=blue,linewidth=2pt]% 1 radian = 57.296 degrees
         {-0.2}{1}{x 360 mul sin}  % sin(2pi*t)
  \psplot[linecolor=red,linewidth=2pt]%
      {-0.2}{1}{x 360 mul 45 add sin} % cos(2pi*t) %La corriente retrasa a la tensión
   \cnode(0.325,0.309017){0.1}{Tension}
   \cnode(0.45,0.309017){0.1}{Corriente}
   \ncline{->}{Tension}{Corriente}%Tension origen de fases
   \naput{$\phi$}
\end{pspicture}

