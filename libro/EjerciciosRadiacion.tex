\chapter{Ejercicios de Radiación Solar}



Calcular las componentes directa y difusa de la radiación solar del
mes de Septiembre (día 261) en un lugar con latitud $\phi=\ang{40}\mathrm{N}$
y con media mensual de irradiación global diaria horizontal $G_{d,m}(0)=\SI{2700}{\Wh\per\meter\squared}$.

\[
K_{t}=0.33\]
\[
F_{d}=0.6271\]
\[
D_{d,m}(0)=\SI{1693.2}{\Wh\per\meter\squared}\]


\[
B_{d,m}(0)=\SI{1006.8}{\Wh\per\meter\squared}\]


Calcular la irradiancia global y la irradiancia difusa en el plano
horizontal 2 horas antes del mediodía del día 261 en un lugar con
latitud $\phi=\ang{40}\mathrm{N}$ y con media mensual de irradiación
global diaria horizontal $G_{d,m}(0)=\SI{2700}{\Wh\per\meter\squared}$.

\[
r_{D}=0.114\]
\[
a=0.6533\]
\[
b=0.893\]
\[
r_{G}=0.163\]
\[
D(0)=\SI{193}{\watt\per\meter\squared}\]


\[
G(0)=\SI{440.1}{\watt\per\meter\squared}\]


Calcular la irradiancia difusa, directa, de albedo y global, en un
generador inclinado $\ang{30}$ y orientado al Sur, 2 horas antes
del mediodía del día 261 en un lugar con latitud $\phi=\ang{40}\mathrm{N}$
y con media mensual de irradiación global diaria horizontal $G_{d,m}(0)=\SI{2700}{\Wh\per\meter\squared}$.

\[
B(\alpha,\beta)=\SI{313.3}{\watt\per\meter\squared}\]


\[
D(\alpha,\beta)=\SI{197.5}{\watt\per\meter\squared}\]


\[
R(\alpha,\beta)=\SI{5.9}{\watt\per\meter\squared}\]


\[
G(\alpha,\beta)=\SI{516.6}{\watt\per\meter\squared}\]


Calcular la irradiación anual efectiva que incide en un generador
orientado al Sur e inclinado $\ang{20}$ en un lugar con latitud $\ang{30}\mathrm{N}$
y una media anual de la irradiación global diaria en el plano horizontal
de $\SI{5250}{\Wh\per\meter\squared}$, suponiendo una suciedad
media.

\[
G_{a}(\beta_{opt})=\SI{5717.3}{\Wh\per\meter\squared}\]
\[
G_{ef,a}(\beta)=\SI{5304.3}{\Wh\per\meter\squared}\]

