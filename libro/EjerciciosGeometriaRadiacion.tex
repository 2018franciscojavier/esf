\chapter{Ejercicios de Geometría y Radiación Solar}


\section{Ángulos Solares}

Calcular el Azimut, Ángulo Cenital y Altura Solar, Duración del Dia
para el:
\begin{itemize}
\item Día del Año: 120, 2 horas después del mediodía, Latitud:
  $37.2\degree\mathrm{N}$
\[
2\cdot\left|\omega_{s}\right|=\SI{13.5}{\hour}\]
\[
\gamma_{s}=55.1\degree\]
\[
\psi_{s}=57.7\degree\]

\item Día del Año: 340, 2 horas después del amanecer, Latitud: $\SI{15}{\degree}\mathrm{S}$
\end{itemize}
\[
2\cdot\left|\omega_{s}\right|=\SI{12.85}{\hour}\]
\[
\gamma_{s}=27.13\degree\]
\[
\psi_{s}=-108.2\degree\]

\begin{itemize}
\item Duración del día 261 del año en las latitudes

\begin{itemize}
\item $\SI{10}{\degree}\mathrm{N}$: $\SI{12.02}{\hour}$
\item $\SI{40}{\degree}\mathrm{N}$: $\SI{12.11}{\hour}$
\item $\SI{70}{\degree}\mathrm{N}$: $\SI{12.37}{\hour}$
\item $\SI{10}{\degree}\mathrm{S}$: $\SI{11.97}{\hour}$
\item $\SI{40}{\degree}\mathrm{S}$: $\SI{11.88}{\hour}$
\item $\SI{70}{\degree}\mathrm{S}$: $\SI{11.63}{\hour}$
\end{itemize}
\item Altura solar en el mediodía del día 25 del año en las latitudes

\begin{itemize}
\item $\SI{10}{\degree}\mathrm{N}$: $\gamma_{s}=60.74\degree$
\item $\SI{40}{\degree}\mathrm{N}$: $\gamma_{s}=30.74\degree$
\item $\SI{10}{\degree}\mathrm{S}$: $\gamma_{s}=80.74\degree$
\item $\SI{40}{\degree}\mathrm{S}$: $\gamma_{s}=69.26\degree$
\end{itemize}
\item Hora Solar cuando la hora oficial son las 12 del mediodía en un día
de verano en un lugar de la península con Longitud $\SI{3}{\degree}\mathrm{O}$.\[
\omega=-33\degree=-\SI{2.2}{\hour}\]

\end{itemize}

\section{Ángulos de Sistemas Fotovoltaicos}

Calcular el ángulo de incidencia para el
\begin{itemize}
\item Día del Año: 120, 2 horas después del mediodía, Latitud: $\SI{37.2}{\degree}\mathrm{N}$;

\begin{itemize}
\item Un sistema estático orientado al Sur y con inclinación de $\SI{30}{\degree}$:
$\theta_{s}=30.33\degree$
\item Un sistema de seguimiento horizontal N-S: $\theta_{s}=17.98\degree$
\item Un sistema de seguimiento acimutal con inclinación a $\SI{35}{\degree}$:
$\theta_{s}=0\degree$
\item Un sistema de seguimiento a doble eje: $\theta_{s}=0\degree$
\end{itemize}
\item Día del Año: 340, 2 horas después del amanecer, Latitud: $\SI{15}{\degree}\mathrm{S}$;

\begin{itemize}
\item Un sistema estático orientado al Sur y con inclinación de $\SI{30}{\degree}$:
$\theta_{s}=75.1\degree$
\item Un sistema de seguimiento horizontal N-S: $\theta_{s}=16.1\degree$
\item Un sistema de seguimiento acimutal con inclinación a $\SI{35}{\degree}$:
$\theta_{s}=27.87\degree$
\item Un sistema de seguimiento a doble eje: $\theta_{s}=0\degree$
\end{itemize}
\end{itemize}



\section{Componentes de irradiación}



Calcular las componentes directa y difusa de la radiación solar del
mes de Septiembre (día 261) en un lugar con latitud $\phi=\ang{40}\mathrm{N}$
y con media mensual de irradiación global diaria horizontal $G_{d,m}(0)=\SI{2700}{\Wh\per\meter\squared}$.

\[
K_{t}=0.33\]
\[
F_{d}=0.6271\]
\[
D_{d,m}(0)=\SI{1693.2}{\Wh\per\meter\squared}\]


\[
B_{d,m}(0)=\SI{1006.8}{\Wh\per\meter\squared}\]


\section{Componentes de irradiancia en el plano horizontal} 
Calcular la irradiancia global y la irradiancia difusa en el plano
horizontal 2 horas antes del mediodía del día 261 en un lugar con
latitud $\phi=\ang{40}\mathrm{N}$ y con media mensual de irradiación
global diaria horizontal $G_{d,m}(0)=\SI{2700}{\Wh\per\meter\squared}$.

\[
r_{D}=0.114\]
\[
a=0.6533\]
\[
b=0.893\]
\[
r_{G}=0.163\]
\[
D(0)=\SI{193}{\watt\per\meter\squared}\]


\[
G(0)=\SI{440.1}{\watt\per\meter\squared}\]


\section{Componentes de irradiancia en el plano del generador} 
Calcular la irradiancia difusa, directa, de albedo y global, en un
generador inclinado $\ang{30}$ y orientado al Sur, 2 horas antes
del mediodía del día 261 en un lugar con latitud $\phi=\ang{40}\mathrm{N}$
y con media mensual de irradiación global diaria horizontal $G_{d,m}(0)=\SI{2700}{\Wh\per\meter\squared}$.

\[
B(\alpha,\beta)=\SI{313.3}{\watt\per\meter\squared}\]


\[
D(\alpha,\beta)=\SI{197.5}{\watt\per\meter\squared}\]


\[
R(\alpha,\beta)=\SI{5.9}{\watt\per\meter\squared}\]


\[
G(\alpha,\beta)=\SI{516.6}{\watt\per\meter\squared}\]


\section{Irradiación anual efectiva} 
Calcular la irradiación anual efectiva que incide en un generador
orientado al Sur e inclinado $\ang{20}$ en un lugar con latitud $\ang{30}\mathrm{N}$
y una media anual de la irradiación global diaria en el plano horizontal
de $\SI{5250}{\Wh\per\meter\squared}$, suponiendo una suciedad
media.

\[
G_{a}(\beta_{opt})=\SI{5717.3}{\Wh\per\meter\squared}\]
\[
G_{ef,a}(\beta)=\SI{5304.3}{\Wh\per\meter\squared}\]

